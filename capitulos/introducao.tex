% ----------------------------------------------------------
% Introdução
% ----------------------------------------------------------
\chapter{Introdução}

Citação com autor no texto, segundo \textcite{araujo2012} isso e tal.

Citação sem citar o autor no texto \cite{araujo2012}.

Citação com autor no texto, segundo  \textcite{ferreira2018topin} isso e tal.

Citação sem citar o autor no texto \cite{ferreira2018topin}.


Citação com autor no texto, segundo  \textcite{guarino1995} isso e tal.

Citação sem citar o autor no texto \cite{guarino1995}.


Citação com autor no texto, segundo  \textcite{de2023thesis} isso e tal.

Citação sem citar o autor no texto \cite{de2023thesis}.


Parte inicial do texto na qual devem constar: a delimitação do assunto tratado (apresentar um resumo do campo teórico em que se localiza o trabalho), os objetivos da pesquisa, a metodologia (como foi feito) e uma justificativa, destacando a relevância do trabalho. Essas informações podem ser feitas em texto corrido (sem subdivisões), porém, se o aluno preferir, poderá dentro do grande tópico “Introdução”, fazer subdivisões para: objetivos, metodologia e justificativa. 

Todo texto deve ser digitado em fonte Arial ou Times New Roman, tamanho 12, inclusive a capa, com exceção das citações com mais de três linhas, notas de rodapé, paginação, dados internacionais de catalogação-na publicação (ficha catalográfica), legendas e fontes das ilustrações e das tabelas, que devem ser em fonte tamanho menor 10 ou 11. O texto deve ser justificado, exceto as referências, no final do trabalho, que devem ser alinhadas à esquerda, espaçamento simples (1,0). 

As páginas devem ser contadas a partir da folha de rosto, mas somente numerada na introdução em diante. 

Todos os autores citados devem ter a referência incluída em lista no final no trabalho. \cite{de2011associaccao}